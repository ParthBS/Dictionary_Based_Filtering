
\documentclass[journal]{IEEEtran}
\usepackage[section]{placeins}
\usepackage{blindtext}
\usepackage{graphicx}
\usepackage{caption}
\usepackage{amsmath}
\usepackage{url}

\ifCLASSINFOpdf
\else
\fi
\hyphenation{Dictionary Based Filtering}


\begin{document}
	%
	% paper title
	\title{Dictionary Based Filtering}
	
	\author{Aatman Dholakia,~\IEEEmembership{1401013},
		Parth Satodiya,~\IEEEmembership{1401056},
		Anuj Shah,~\IEEEmembership{1401084},
		Vishal Raiyani,~\IEEEmembership{1401094}}
	
	
	
	
	% make the title area
	\maketitle
	
	
	\begin{abstract}
		%\boldmath
		Main objective is to denoise image efficiently using previous N x N patches of noisy images and it's filtered images and find nearest patch which compares to the original image from this dictionary and then compare it's complexity and efficiency with classical convolution based filtering. Also we compare efficiency between different image patch sizes.
		
	\end{abstract}
	\begin{IEEEkeywords}
		Classical convolution, low pass filter, dictionary learning, N x N patches
	\end{IEEEkeywords}
	
	
	\IEEEpeerreviewmaketitle
	
	
	
	\section{\textbf{Introduction}}
    First we take some training images of dimensions M x N and filtered image is obtained by classical convolution. Now we divide this image into patches each of dimensions P x Q  where $P, Q \le M, N$ and store it in the dictionary. In the dictionary the key is noisy part of the image and the value is filtered part of the image. Now test image is taken and divided into patches and corresponding value is obtained by iterating the keys. If key is not found then add this new key - value pair in dictionary database.
    

	\section{\textbf{Flow chart}}
    \begin{minipage}{\linewidth}
		\centering
		\includegraphics[width = 90mm]{final.png}
		\captionof{figure}{Flow chart \label{overflow}}
	\end{minipage} 
		
	
	\section{\textbf{Methodology}}
	\begin{enumerate}
	    \item \textbf{Low pass filtering}\\
	    The most basic of filtering operations is called "low-pass". A low-pass filter, also called a "blurring" or "smoothing" filter, averages out rapid changes in intensity. The simplest low-pass filter just calculates the average of a pixel and all of its eight immediate neighbors. The result replaces the original value of the pixel. The process is repeated for every pixel in the image.\\
	    \item \textbf{Classical convolution}\\
        \begin{minipage}{\linewidth}
    		\centering
    		\includegraphics[width = 70mm]{conv.png}
    		\captionof{figure}{Linear convolution \label{overflow}}\\
	    \end{minipage}     	    
	    \item \textbf{Salt and pepper noise reduction}\\
	    Salt-and-pepper noise is a form of noise sometimes seen on images. It presents itself as sparsely occurring white and black pixels.\\
	    For reducing either salt noise or pepper noise, but not both, a contraharmonic mean filter can be effective.\\
	    It is done through low pass filtering.
	    \item \textbf{K - SVD Decomposition Dictionary Learning}\\
	    K-SVD method learns an over-complete dictionary from an
input image via solving the following minimization model:
        \begin{minipage}{\linewidth}
    		\centering
    		\includegraphics[width = 70mm]{eq_1.png}
    		\captionof{figure}{K - SVD \label{overflow}}
	    \end{minipage} \\
	


 where $ gi \subset R_n$ is the collection of image patches after vectorization. $D = [d1, . . . , dk] \subset R $, n×k with k > n is the unknown over-complete dictionary.
	    
	\end{enumerate}
	
	\section{\textbf{Simulation results}}
	    \begin{minipage}{\linewidth}
    		\centering
    		\includegraphics[width = 70mm]{crop.png}
    		\captionof{figure}{Original and noisy image \label{overflow}}
	    \end{minipage} 
	
	\newpage
	\section{\textbf{Further improvements}}
	\begin{enumerate}
	    \item Searching algorithm
	    \item Matching algorithm\\
	    \begin{minipage}{\linewidth}
    		\centering
    		\includegraphics[width = 70mm]{for.png}
    		\captionof{figure}{Forbenius Law \label{overflow}}
	    \end{minipage} 
	
	    \item Comparision between classical and dictionary based algorithm for different number of patches.
	\end{enumerate}
	
	\begin{thebibliography}{1}
	\bibitem{IEEEhowto:kopka}
	“Digital Image Processing”, JAYARAMAN
	\bibitem{IEEEhowto:kopka}
	"Median filter", En.wikipedia.org, 2017. [Online]. Available:
	\bibitem{IEEEhowto:kopka}\url{https://en.wikipedia.org/wiki/Median_filter}. [Accessed: 03- Mar- 2017].\\
	\bibitem{IEEEhowto:kopka}
	\url{https://lear.inrialpes.fr/people/mairal/tutorial_iccv09/tuto_part2.pdf, 2017. Web. 6 Mar. 2017.}\\
	\bibitem{IEEEhowto:kopka}
	\url{https://pdfs.semanticscholar.org/45b0/676a6b8236cefaaeac0faabea159369fdb65.pdf, 2017. Web. 7 Mar. 2017}\\
	\bibitem{IEEEhowto:kopka}
	\url{ http://www.dspguide.com/ch6/2.htm}\\
	\bibitem{IEEEhowto:kopka}
	\url{ https://en.wikipedia.org/wiki/Salt-and-pepper_noise}
	\bibitem{IEEEhowto:kopka}
	\url{ http://www.bogotobogo.com/Matlab/Matlab_Tutorial_Digital_Image_Processing_6_Filter_Smoothing_Low_Pass_fspecial_filter2.php
}
	
	\end{thebibliography}
	
	
	\ifCLASSOPTIONcaptionsoff
	\newpage
	\fi
	
\end{document}


